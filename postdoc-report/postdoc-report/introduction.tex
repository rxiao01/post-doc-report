\chapter{Introduction}

\section{Context and objectives}

In our daily lives, we have seen a variety of road surfaces, including bituminous (or asphalt) roads, concrete roads, earthen roads, gravel roads, and murram roads. 
Among these, bituminous roads are the most commonly seen in our daily transportation due to their practicality, wear resistance and robustness.
In the design and operation of mobility systems, many challenges, e.g. climate imperatives, the rise of active modes of transport, and the arrival of autonomous vehicles, require us to rethink a number of practices.
In this context, the optical properties of road surface play a fundamental role in optimizing the energy consumption of public lighting installations~\cite{2021_Muzet}, reduce light pollution~\cite{2019_Muzet}, control urban temperature~\cite{2018_Rossi}, and consider perception of road markings~\cite{2020_Burghardt}.
In favor of these issues, it becomes necessary to know the optical properties of road surface or be able to anticipate them when creating a surfacing or predicting their evolution over time.
Consequently, this requires conducting a significant number of simulations or measurements.


When discussing optical properties of road surface, both the Bidirectional Reflectance Distribution Function (BRDF) and solar albedo are essential.
The former provides detailed directional information, describing the angular distribution of reflected light based on the angle of incoming light and the viewing angle~\cite{2008_Jarosz}. 
It captures how surface roughness, texture, and material composition affect light reflection, affecting the visibility of the road and markings~\cite{2010_Roser}.
The latter gives an overall measure of reflected energy, representing the total fraction of incoming solar energy reflected by the road surface, without regard to direction~\cite{2011_KUSHARI}.
Road surface with higher solar albedo reduces heat absorption by reflecting more sunlight back into the atmosphere, lowering cooling demands in nearby buildings and contributing to mitigate the damage on road materials. 


To know these two optical properties, it requires to perform extensive measurements or simulations.
The former are usually costly, not only in terms of instrumentation but also in the time required to measure it.
Although launching simulations can address these limits, it needs to find an appropriate reflectance model which perfectly fit the measured.
As known, roads are usually composed of multiple materials, such as bitumen, different aggregates, filler.






\section{Mission}
To achieve the objective, the mission follows 5 steps:
\begin{enumerate}
    \item Build several single-component road samples and measure their BRDF;
    \item Fit the measured BRDF of each single-component road sample using existing BRDF models;
    \item Build several multi-component road samples and measure their BRDF;
    \item Explore and develop the method to combine the BRDFs of each single-component to fit the measurement of multi-component road samples;
    \item Estimate the solar albedo using BRDF.
\end{enumerate}

\section{Report organization}
This report is structured into the following chapters: 
% In chapter 2, it provides the detailed definition and properties of BRDF, investigates several BRDF models used in road surface, and its measurement set-up.
% Then it introduce the solar albedo including black sky albedo, white sky albedo and blue sky albedo and the link between solar albedo and BRDF.
% In chapter 3, it summarizes the previous work on the simulation and measurement of single-component (e.g. different rocks), and investigates method about how to fit the measurement using BRDF models from literature study. Then it present measured and simulated results for each single-component road sample.
% In chapter 4, it investigates the existing method of mix different BRDF models, and propose new method to combine all BRDF models of each single-component to fit the measurement. Then it presents the corresponding results for each multi-component road sample.
% In chapter 5, it concludes this report and provides some new perspectives for future work.