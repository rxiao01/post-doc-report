\chapter{Optical properties of road surface}\label{ch:optical-properties-of-road-surface}
\label{ch: optical properties}


%%%%%%%%%%%%%%%%%%%%%%%%%%%%%%%%%%%%%%%%%%%%%%%%%%%%%%%%%%%%%%%%%%%%%%%%%%%%%%%%%%%%%%%%%%%%%%%%%%%%%%%%%%%%%%%%%%%%
\section{Basic concepts}

To provide a clear understanding of the optical properties of road surface, the fundamental terminologies are introduced in this section.

%%%%%%%%%%%%%%%%%%%%%%%%%%%%%%%%%%%%%%%%%%%%
\subsection{Radiometric quantities}

Radiometric quantities provide a way to measure and describe the behavior of radiant energy in terms of light interaction with road surface.
They include flux, radiant intensity, irradiance and radiance.

%%%%%%%%%%%%%%%%%%%%%%%%%%%%%%%
\textbf{Flux $F$}, is the most fundamental quantity in radiometry~\cite{2008_Jarosz}.
It is also called radiant power, which measures the amount of light that hits a surface over a finite area from all directions per unit time.
For a given amount of radiated energy $Q$ at a time duration $t$, the flux $F$ is expressed as:
\begin{equation}
    F(t) = \frac{\mathrm{d}Q}{\mathrm{d}t}
\end{equation}
As $Q$ is expressed in Joules ($J$), the unit of $F$ is watts ($W = J\cdot s^{-1}$)


%%%%%%%%%%%%%%%%%%%%%%%%%%%%%%%
\textbf{Radiant intensity $I$}, is a correlated measure of flux.
It represents the intensity of flux per unit solid angle which is propagate towards some specific direction $(\theta, \varphi)$ toward the the infinitesimal solid angle $\mathrm{d}\overrightarrow{\omega}$~\cite{2022_Hebert}.
Thus, it can be expressed in terms of flux:
\begin{equation}
    I(t, \theta, \varphi) = \frac{\mathrm{d}F(t)}{\mathrm{d}\overrightarrow{\omega}}%
    =  \frac{\mathrm{d}F(t)}{\sin \theta \mathrm{d}\theta \mathrm{d} \varphi}
\end{equation}
Notice that the unit of solid angle is the steradian $[sr]$, so the unit of radiant intensity is $[W \cdot {sr}^{-1}]$.


%%%%%%%%%%%%%%%%%%%%%%%%%%%%%%%
\textbf{Irradiance $E$}, is another correlated measure of flux in terms of surface area.
Different from radiant intensity, it captures the integration over the entire hemisphere $\Omega$ of the incident light arriving at a unit surface $\mathrm{d}s(p)$ centered on the point $p$.
Essentially, it measure the amount of radiated energy strike a unit area per unit time:
\begin{equation}
    E(p, t) = \frac{\mathrm{d}F(t)}{\mathrm{d}s(p)}
\end{equation}
Its unit is $[W \cdot m^{-2}]$


%%%%%%%%%%%%%%%%%%%%%%%%%%%%%%%
\textbf{Radiance $L$}, measures the amount of incident light arriving at a unit surface $\mathrm{d}s(p)$ centered on $p$ from a unit solid angle $\mathrm{d}\overrightarrow{\omega}$ per unit time $t$.
It can be considered as a correlated measure of radiant intensity per unit area or irradiance per unit solid angle:
\begin{equation}
    \label{eq_irradiance_radiance_intensity}
    L(p, t, \theta, \varphi) = \frac{\mathrm{d}^2 F(t)}{\left| \overrightarrow{\omega} \cdot \overrightarrow{n} \right| \mathrm{d} \overrightarrow{\omega} \mathrm{d}s(p)}%
    = \frac{\mathrm{d}I(t, \theta, \varphi)}{\cos \theta \mathrm{d}s(p)}%
    = \frac{\mathrm{d} E(p, t)}{\sin \theta  \mathrm{d}\theta \mathrm{d} \varphi}%
\end{equation}
The product term $\cos \theta \mathrm{d}s(p)$ represents the projection of the unit surface $\mathrm{d}s(p)$ onto the direction $\overrightarrow{\omega}$.
According to its definition, the unit of $L(p, t, \theta, \varphi)$ is $[W \cdot m^{-2} \cdot {sr}^{-1}]$.

\subsection{Spectral radiometry}

In the previous section, the definitions of radiometric quantities are given without considering the wavelength.
In order to describe the spectral distribution of a radiation, the spectral radiometric quantities are introduced, including spectral flux, spectral intensity, spectral irradiance and spectral radiance.

\textbf{Spectral flux $F_\lambda$}, is defined as the flux per unit wavelength and expressed in $[W \cdot {nm}^{-1}]$:
\begin{equation}
    F_\lambda(t) = \frac{\mathrm{d}F(t)}{\mathrm{d}\lambda}
\end{equation}

\textbf{Spectral intensity $I_\lambda$}, is the radiant intensity per unit wavelength and expressed in $[W \cdot {sr}^{-1} \cdot {nm}^{-1}]$:
\begin{equation}
    I_\lambda(t, \theta, \varphi)  =  \frac{\mathrm{d}I(t, \theta, \varphi)}{\mathrm{d}\lambda}
\end{equation}

\textbf{Spectral irradiance $E_\lambda$}, is irradiance per unit wavelength and expressed in $[W \cdot m^{-2} \cdot {nm}^{-1}]$:
\begin{equation}
    E_\lambda(p, t) = \frac{\mathrm{E(p, t)}}{\mathrm{d}\lambda}
\end{equation}

\textbf{Spectral radiance $L_\lambda$}, is defined as the radiance per unit wavelength and expressed in $[W \cdot m^{-2} \cdot {sr}^{-1} \cdot {nm}^{-1}]$:
\begin{equation}
    L_\lambda(p, t, \theta, \varphi) = \frac{\mathrm{d}L(p, t, \theta, \varphi)}{\mathrm{d}\lambda}
\end{equation}

The measurement of spectral quantities are conducted with instruments such as spectrophotometers which analyze the radiation in adjacent and narrow spectral bands~\cite{2022_Hebert}.
When analyzing radiation over a broader wavelength range $[\lambda_1, \lambda_2]$, the total measured flux can be calculated by integrating the spectral flux density across the specified waveband.
This process accounts for the contribution of all individual wavelengths within the range, providing a cumulative representation of the flux.
The computation is expressed as:
\begin{equation}
    F(\lambda_1, \lambda_2) = \int_{\lambda_1}^{\lambda_2} F_\lambda \mathrm{d}\lambda
\end{equation}

% \begin{table}[h]
%     \centering
%     \caption{Radiometric measurements and their photometric analogs}
%     \begin{tabular}{llll}
%         \hline
%         Radiometric    & Unit         & photometric        & Unit                          \\
%         \hline
%         Radiant energy & joule($J$)   & Luminous energy    & talbot (T)                    \\
%         Radiant flux   & watt ($W$)   & Luminous flux      & lumen ($lm$)                  \\
%         Intensity      & $W/sr$       & Luminous intensity & $lm/sr = candela (cd)$        \\
%         Irradiance     & $W/m^2$      & Illuminance        & $lm/m^2 = lux$                \\
%         Radiance       & $W/(m^2 sr)$ & Luminance          & $lm / (m^2sr) = cd/m^2 = nit$ \\
%         \hline
%     \end{tabular}
% \end{table}

% \subsection{Photometry}

%%%%%%%%%%%%%%%%%%%%%%%%%%%%%%%%%%%%%%%%%%%%%%%%%%%%%%%%%%%%%%%%%%%%%%%%%%%%%%%%%%%%%%%%%%%%%%%%%%%%%%%%%%%%%%%%%%%%
\section{Spectral BRDF}
\label{sec: spectral brdf}

The BRDF is first introduced by Fred Nicodemus in 1965~\cite{1965_Nicodemus}, which is rigorously defined as the ratio of reflected radiance in a given viewing direction to the incident irradiance from a specified light source direction, at a given surface point $p$.
It is typically expressed in units of $sr^{-1}$.
However, the BRDF alone does  not capture the full complexity of how surfaces reflect light across different wavelengths.
To gain a more comprehensive understanding of these interactions, we turn to the spectral BRDF which is an extension of the traditional BRDF~\cite{2022_Hebert}.
It provides a wavelength-dependent function that characterizes how surfaces reflect light at specific incident and reflected angles at a given surface point $p$, across a narrow spectral bandwidth $\Delta \lambda$.
Mathematically, it is expressed as:
\begin{equation}
    f_r(p, \lambda, \theta_i, \varphi_i, \theta_r, \varphi_r) = \frac{\mathrm{d}{L(p, \lambda, \theta_r, \theta_r)}}{\mathrm{d}E(p, \lambda, \theta_i, \varphi_i)},
\end{equation}
where:
\begin{itemize}
    \item $\lambda$ is the wavelength of light source;
    \item $\theta_i$ and $\varphi_i$ are the zenith and azimuth angle of the unit incident direction $\overrightarrow{\omega_i}$:
          \[
              \overrightarrow{\omega_i} = (x_i, y_i, z_i)= (\sin\theta_i \cos \varphi_i, \sin\theta_i \sin\varphi_i, \cos\theta_i)
          \]
    \item $\theta_r$ and $\varphi_r$ are the zenith and azimuth angle of the unit reflected direction $\overrightarrow{\omega_r}$:
          \[
              \overrightarrow{\omega_r} = (x_r, y_r, z_r)= (\sin\theta_r \cos \varphi_r, \sin\theta_r \sin\varphi_r, \cos\theta_r)
          \]
    \item $\mathrm{d}L(p, \lambda, \theta_r, \theta_r)$ is the spectral reflected radiance for waveband $\Delta \lambda$;
          \[
              \mathrm{d}L(p, \lambda, \theta_r, \theta_r) = \int_{\Delta \lambda} \mathrm{d}L_\lambda(p, \theta_r, \varphi_r) \mathrm{d}\lambda
          \]
    \item $\mathrm{d}E(p, \lambda, \theta_i, \varphi_i)$ is the spectral incident irradiance for waveband $\Delta \lambda$.
          \[
              \mathrm{d}E(p, \lambda, \theta_i, \theta_i) = \int_{\Delta \lambda} \mathrm{d}E_\lambda(p, \theta_i, \varphi_i) \mathrm{d}\lambda
          \]
\end{itemize}

Replacing the term $\mathrm{d}E(p, \lambda, \theta_i, \varphi_i)$ according to Equation~\eqref{eq_irradiance_radiance_intensity}, it can be defined in terms of incident radiance $L_i(p, \lambda, \theta_i, \varphi_i)$:
\begin{equation}
    f_r(p, \lambda, \theta_i, \varphi_i, \theta_r, \varphi_r) = \frac{\mathrm{d}{L(p, \lambda, \theta_r, \theta_r)}}{\cos\theta_i L_i(p, \lambda, \theta_i, \varphi_i) \mathrm{d} \theta_i \mathrm{d} \varphi_i}
\end{equation}
As both radiance and irradiance take into account the wavelength, the unit of spectral BRDF is still $[sr^{-1}]$.

\textbf{Ideal diffuse reflection}

One of the simplest and most widely used BRDF models is the Lambertian reflector, which is based on the assumption of a perfectly diffuse surface~\cite{2022_Hebert}.
It has an angle-independent BRDF, proportional to their spectral albedo $\rho_d$ for a given wavelength $\lambda$:
\begin{equation}
    \label{eq_Lambertian}
    f_{r(Lambertian)}(p, \lambda, \theta_i, \varphi_i, \theta_r, \varphi_r) = \frac{\rho_d}{\pi}
\end{equation}

\textbf{Ideal specular reflection}

In contrast to diffuse reflection, the perfect specular BRDF describes light incoming a given direction is reflected in a single direction following the law of reflection~\cite{2012_Montes}:
\begin{equation}
    \label{eq_specular}
    f_{r(specular)}(p, \lambda, \theta_i, \varphi_i, \theta_r, \varphi_r) = F_r(\lambda, \theta_i) \frac{\delta(\overrightarrow{\omega_r} - \overrightarrow{\omega_r^\prime})}{\cos\theta_r},
\end{equation}
where:
\begin{itemize}
    \item $\overrightarrow{\omega_r^\prime}$ is the mirror direction which is symmetric to the incoming direction $\overrightarrow{\omega_i}$:
          \[
              \overrightarrow{\omega_r^\prime} = 2(\overrightarrow{\omega_i} \cdot \overrightarrow{n})\overrightarrow{n} -\overrightarrow{\omega_i} = 2 \cos \theta_i \overrightarrow{n} - \overrightarrow{\omega_i} =(-\sin\theta_i \cos \varphi_i, -\sin\theta_i \sin\varphi_i, \cos\theta_i)
          \]
          So we can derive its corresponding zenith and azimuth angle $(\theta_r^\prime, \varphi_r^\prime) = (\theta_i, \varphi_i +\pi)$.
    \item $\delta(\overrightarrow{\omega_r} - \overrightarrow{\omega_r^\prime})$ is the delta dirac function:
          \[
              \delta(\overrightarrow{\omega_r} - \overrightarrow{\omega_r^\prime}) = \begin{cases}
                  \infty, & \mbox{if $\overrightarrow{\omega_r} = \overrightarrow{\omega_r^\prime}$} \\
                  0,      & \mbox{otherwise}
              \end{cases}
          \]
          Notice that Its integral over all directions equals $1$:
          \[
              \int_{\Omega} \delta(\overrightarrow{\omega_r} - \overrightarrow{\omega_r^\prime}) \mathrm{d} \overrightarrow{\omega_r} =1
          \]
          This ensures energy conservation in the reflection process.
    \item $F_r(\lambda, \theta_i)$ is the Fresnel reflectance of unpolarized light for a given wavelength $\lambda$ following as:
          \begin{equation}
              \label{eq_fresnel}
              F_r(\lambda, \theta_i) = \frac{1}{2}\left(r_{\parallel}(\lambda, \theta_i) + r_{\perp}(\lambda, \theta_i)\right)
          \end{equation}
          $r_{\parallel}(\lambda, \theta_i)$ and $r_{\perp}(\lambda, \theta_i)$ are the conventional Fresnel coefficient~\cite{2005_Lazanyi} for a given wavelength $\lambda$, respectively for parallel and perpendicular polarized light.
          They are given as below while considering different interfaces:
          \begin{itemize}
              \item At the interface of two dielectric media:
                    \[
                        r_{\parallel}(\lambda, \theta_i) = \left[\frac{\eta_t \cos\theta_i - \eta_i\cos\theta_t}{\eta_t\cos\theta_i + \eta_i\cos\theta_t}\right]^2
                    \]
                    \[
                        r_{\perp}(\lambda, \theta_i) = \left[\frac{\eta_i \cos\theta_i - \eta_t\cos\theta_t}{\eta_i\cos\theta_i + \eta_t\cos\theta_t}\right]^2
                    \]
                    Where $\eta_i, \eta_t$ are the refraction indices for the incident and transmitted media for a given wavelength $\lambda$.
                    $\theta_t$ is the transmitted angle between the normal and transmitted direction:
                    \[
                        \sin\theta_t = \frac{\eta_i * \sin\theta_i}{\eta_t}
                    \]
              \item At the boundary between a conductor and a dielectric medium:
                    \[
                        r_{\parallel}(\lambda, \theta_i) = r_\perp(\lambda, \theta_i) \frac{\cos^2\theta_i(a^2 + b^2) - 2a\cos\theta_i\sin^2\theta_i + \sin^4\theta_i}%
                        {\cos^2\theta_i(a^2 + b^2) + 2a\cos\theta_i\sin^2\theta_i + \sin^4\theta_i}
                    \]
                    \[
                        r_{\perp}(\lambda, \theta_i) = \frac{a^2 + b^2 -2a\cos\theta_i + \cos^2\theta_i}{a^2 + b^2 +2a\cos\theta_i + \cos^2\theta_i}
                    \]
                    Where:
                    \[
                        \begin{array}{lll}
                            2a^2                        & = \sqrt{(\eta^2 -k^2 -\sin^2\theta_i)^2 + 4\eta^2 k^2 } + (\eta^2 - k^2 -\sin^2\theta_i) \\
                            2b^2                        & = \sqrt{(\eta^2 -k^2 -\sin^2\theta_i)^2 + 4\eta^2 k^2 } - (\eta^2 - k^2 -\sin^2\theta_i) \\
                            a^2 + b^2                   & = \sqrt{(\eta^2 -k^2 -\sin^2\theta_i)^2 + 4\eta^2 k^2 }                                                    \\
                            \eta + ik & = \frac{\eta_t + ik_t}{\eta_i+ ik_i}
                        \end{array}
                    \]

          \end{itemize}

          Computing the exact Fresnel term can be highly time-consuming.
          To mitigate this, Schlick proposed an approximation in 1994~\cite{1994_Schlick}, which has become widely adopted in computer graphics~\cite{2005_Lazanyi,2021_Majercik}.
          This approximation is following as below when considering different interfaces:
          \begin{itemize}
              \item At the interface between 2 dielectric mediums:
                    \begin{equation}
                        \label{eq_fresnel_dielectric_schlick}
                        F_r(\lambda, \theta_i) = F_0(\lambda) + (1 - F_0(\lambda))(1-\cos\theta_i)^5
                    \end{equation}
                    The term $F_0(\lambda)$ is the reflection coefficient for light incoming parallel to the normal:
                    \[
                        F_0(\lambda) = \left(\frac{\eta_i - \eta_t}{\eta_i + \eta_t}\right)^2
                    \]
              \item At the interface between air/vacuum and a conductor:
                    \begin{equation}
                        \label{eq_fresnel_conductor_schlick}
                        F(\lambda, \theta_i) = \frac{(\eta_t -1)^2 + 4\eta_t(1-\cos\theta_i)^5 + k_t^2 }{(\eta_t +1)^2 + k_t^2}
                    \end{equation}
                    where $\eta_t $ is the the index of refraction for the conductor and $k_t$ is the absorption coefficient of the conductor.
          \end{itemize}
          Notice that when this approximation is used in a microfacet-based BRDF model for describing specular reflection, the angle $\theta_i$ corresponds to the local incident angle $\theta_i^\prime$.
          In geometry, it is the angle between the half-vector direction between half-vector direction $\overrightarrow{\omega_h} = \frac{\overrightarrow{\omega_i} + \overrightarrow{\omega_r}}{|\overrightarrow{\omega_i} + \overrightarrow{\omega_r}|}$ and the incident direction $\overrightarrow{\omega_i}$.

\end{itemize}

%%%%%%%%%%%%%%%%%%%%%%%%%%%%%%%%%%%%%%%%%%%%%%%%%%%%%%%%%%%%%%%%%%%%%%%%%%%%%%%%%%%%%%%%%%%%%%%%%%%%%%%%%%%%%%%%%%%%
\subsection{Properties}

Similar to traditional BRDF, a physically plausible spectral BRDF also has the below properties:
\begin{itemize}
    \item \textbf{Energy conservation}

          A fundamental principle that must be adhered to by all BRDFs, including spectral BRDFs, is energy conservation.
          This ensures that the amount of light reflected by a surface does not exceed the amount of light incident upon it.
          Mathematically, this is often represented as:
          \[
              \int_{\varphi_r = 0}^{2\pi} \int_{\theta_r =0}^{\frac{\pi}{2}} f_r(p, \lambda, \theta_i, \varphi_i, \theta_r, \varphi_r) \cos(\theta_r) \sin(\theta_r) \mathrm{d}\theta_r \mathrm{d}\varphi_r \le 1
          \]
    \item \textbf{Reciprocity}

          The Helmholtz Reciprocity Rule states that the reflection characteristics should remain unchanged if the directions of light incidence and observation are swapped:
          \[
              f_r(p, \lambda, \theta_i, \varphi_i, \theta_r, \varphi_r)  = f_r(p, \lambda,  \theta_r, \varphi_r, \theta_i, \varphi_i)
          \]
          It ensures that light behaves consistently in all directions.


    \item \textbf{Non-negative}

          The spectral BRDF is always non-negative, ensuring that the reflected radiance is physically meaningful:
          \[
              f_r(p, \lambda, \theta_i, \varphi_i, \theta_r, \varphi_r) \ge 0
          \]
\end{itemize}

%%%%%%%%%%%%%%%%%%%%%%%%%%%%%%%%%%%%%%%%%%%%%%%%%%%%%%%%%%%%%%%%%%%%%%%%%%%%%%%%%%%%%%%%%%%%%%%%%%%%%%%%%%%%%%%%%%%%
\subsection{Isotropic and anisotropic}

As spectral BRDF is an extension of traditional BRDF, it can be classified into two categories based on whether they exhibit rotational symmetry or not~\cite{2012_Montes, 2015_Filip}: isotropic and anisotropic.
A material is considered isotropic when its reflectance remains constant for a fixed view and illumination, regardless of the rotation of the material around its normal.
Mathematically, it can be expressed in term of the azimuth angle difference between incident direction and reflected direction:
\[
    f_r(p, \lambda, \theta_i, \varphi_i, \theta_r, \varphi_r) = f_r(p, \lambda, \theta_i, \theta_r, \varphi_r - \varphi_i)
\]

In contrast, materials whose reflectance is not constant are considered anisotropic.


%%%%%%%%%%%%%%%%%%%%%%%%%%%%%%%%%%%%%%%%%%%%%%%%%%
\section{Solar albedo}

Another important optical property of road surface, solar albedo (also called solar reflectance) is introduced and discussed in this section.


%%%%%%%%%%%%%%%%%%%%%%%%%%%%%%%%%%%%%%%%%%%%%%%%%%
\subsection{Types of albedo}

Solar albedo of a given surface is defined as the ratio of upward and downward radiation flux~\cite{2015_Qu,2019_Chen}.
This ratio can vary between 0 and 1, where 0 indicates no reflection and 1 signifies complete reflection.
Surfaces with higher albedo values reflect more sunlight, leading to cooler temperatures, while those with lower values absorb more radiation, resulting in higher temperatures.
As such, it becomes an important factor affecting the temperature of a sunlit surface and that of near-surface ambient air temperature.

There are three primary types of albedo~\cite{2015_Qu} that are commonly discussed, each with distinct implications for environmental and climatic conditions.

\begin{enumerate}
    \item \textbf{Black sky albedo}

          When the surface is illuminated with ideal directional radiation, the surface albedo is called black-sky albedo or directional-hemispherical reflectance:
          \begin{equation}
              \label{eq_black_sky_albedo}
              \alpha_{black-sky}(\lambda, \theta_i, \varphi_i) = \frac{\int_{0}^{\pi/2} \int_{0}^{2\pi} L_\lambda(\theta_r, \varphi_r) \mathrm{d} \theta_i \mathrm{d} \varphi_i}{E_\lambda(\theta_i, \varphi_i)}
          \end{equation}
          Where $L_\lambda$ is the reflected radiance in the direction $\overrightarrow{\omega_r}$, and $E_\lambda$ is the incident irradiance from the direction $\overrightarrow{\omega_i}$.

    \item \textbf{White sky albedo}

          When the surface is illuminated with ideal diffuse radiation, the surface albedo is called white-sky albedo or bi-hemispherical reflectance:
          \begin{equation}
              \label{eq_white_sky_albedo}
              \alpha_{white-sky}(\lambda) = \frac{\int_{0}^{\pi/2} \int_{0}^{2\pi}L_\lambda(\theta_r, \varphi_r) \mathrm{d} \theta_r \mathrm{d} \varphi_r}{E_\lambda}
          \end{equation}
          Different from the term $E_\lambda$ in black-sky albedo, here it represents the incident irradiance from all directions into the hemisphere.

    \item \textbf{Blue sky albedo}

          In fact, the solar albedo we measured is usually under natural daylight illumination, including both directional and diffuse radiation.
          In this case, the surface albedo is called as blue-sky albedo, which can be approximately expressed as a linear combination of black-sky and white-sky albedo:
          \begin{equation}
              \label{eq_blue_sky_albedo}
              \alpha_{blue-sky}(\lambda, \theta_i, \varphi_i) \thickapprox (1 - D(\tau, \lambda)) \alpha_{black-sky}(\lambda, \theta_i, \varphi_i) + D(\tau, \lambda) \alpha_{white-sky}(\lambda)
          \end{equation}
          Where $D(\tau, \lambda)$ gives the fraction of the diffuse radiation, varying with the aerosol optical wavelength $\tau$ and wavelength $\lambda$.
\end{enumerate}

For a given waveband, its corresponding surface albedo can be estimated using the following equation:
\begin{equation}
    \label{eq_albedo_broadband}
    \alpha(\theta_i, \varphi_i) = \frac{\int_{\lambda_1}^{\lambda_2} E_\lambda(\theta_i, \varphi_i) \alpha_\lambda(\theta_i, \varphi_i) \mathrm{d}\lambda}{\int_{\lambda_1}^{\lambda_2} E_\lambda(\theta_i, \varphi_i) \mathrm{d}\lambda}
\end{equation}

%%%%%%%%%%%%%%%%%%%%%%%%%%%%%%%%%%%%%%%%%%%%%%%%%%
\subsection{The link between BRDF and solar albedo}
According to the definition of BRDF, black-sky albedo and white sky albedo, we can find the link between them.
The black-sky albedo can be derived by integrating BRDF over the viewing hemisphere:
\begin{equation}
    \label{eq_brdf_black_albedo}
    \alpha_{black-sky}(\lambda, \theta_i, \varphi_i) = \int_{\varphi_r = 0}^{2\pi} \int_{\theta_r = 0}^{\frac{\pi}{2}} f_r(\lambda, \theta_i,\varphi_i, \theta_r, \varphi_r) \cos\theta_r \sin\theta_r \mathrm{d}\theta_r \mathrm{d}\varphi_r
\end{equation}
Similarly, the white-sky albedo can be derived by integrating BRDF over the viewing hemisphere and incident hemisphere:
\begin{equation}
    \label{eq_brdf_white_albedo}
    \alpha_{white-sky}(\lambda) = \frac{1}{\pi}\int_{\varphi_i = 0}^{2\pi} \int_{\theta_i = 0}^{\frac{\pi}{2}}%
    \int_{\varphi_r = 0}^{2\pi} \int_{\theta_r = 0}^{\frac{\pi}{2}}% 
    f_r(\lambda, \theta_i,\varphi_i, \theta_r, \varphi_r) %
    \cos\theta_i \sin\theta_i%
    \cos\theta_r \sin\theta_r%
    \mathrm{d}\theta_i \mathrm{d}\varphi_i \mathrm{d}\theta_r \mathrm{d}\varphi_r
\end{equation}

For ideal diffuse reflection, we can conclude that the black-sky albedo and the white-sky albedo are equivalent to the albedo $\rho$ used in Equation~\ref{eq_Lambertian} by computing these two equations.








